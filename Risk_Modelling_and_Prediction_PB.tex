\documentclass[a4paper,12pt]{article}
\usepackage[magyar]{babel}
%\usepackage{blindtext}
\usepackage{graphicx}
%\usepackage{subfig}
\usepackage{float}
\usepackage{booktabs}
\usepackage{pdfpages}
\linespread{1.5}
\renewcommand \thesection{\Roman{section}}
\renewcommand \thesubsection{\Roman{section}.\arabic{subsection}}
%\usepackage{floatrow}
\usepackage{adjustbox,lipsum}
\usepackage{caption}
\usepackage{cite}
\usepackage{apacite}
\usepackage{natbib}
\usepackage{amsmath,amsthm,amsfonts,amssymb,mathtools}
\usepackage{hyperref}
\hypersetup{
    colorlinks=true,
    linkcolor=cyan,
    filecolor=magenta,   
   citecolor=blue,   
    urlcolor=cyan,
    }
\urlstyle{same}
\usepackage{cleveref}
%\usepackage[nameinlink,noabbrev]{cleveref}
\linespread{1.5}
\usepackage{geometry}
%bayes
\usepackage{lmodern}
\usepackage{mathrsfs}
\usepackage{dsfont}
\usepackage{accents}
\usepackage{dutchcal}
\DeclareMathOperator*{\plim}{plim}
\usepackage{caption}

\DeclareMathOperator*{\E}{\mathbb{E}}
\newcommand{\probP}{\text{I\kern-0.15em P}}

 \geometry{
 a4paper,
 total={297mm,210mm},
 left=25mm,
 right=25mm,
 top=25mm,
 bottom=25mm,
 }
\usepackage[utf8]{inputenc}
%\pagenumbering{roman}
\setcounter{page}{2}

%%%%%%%%%%%%%%%%%%%%%%%%%%%%%%%%%%%%%%%%%%%%%%%%%%%%%%%%%%%%%%%%%%%%%%%%%
\begin{document}

\begin{titlepage}

   \begin{center}
       \vspace*{1cm}
        \textbf{\Large Time series momentum and funding liquidity in uncertainty} \\
       \vspace{0.5cm}
       \vspace{1.5cm}
       \textbf{Plesz Boldizsár}
       \vfill
       %\vspace{0.8cm}
       Budapesti Corvinus Egyetem\\
       Gazdaság- és pénzügy-matematikai elemzés\\
       Gazdaság-matematika specializáció \\
       2023 \\
	
    \end{center}
\end{titlepage}

\newpage
\tableofcontents
\newpage

\section{1. Feladat}
\subsection{Adatok}
Az elemzéshez két árupiaci eszközt választottam, és az ezekhez tartozó határidős kontraktusokkal számoltam. 
Az egyik az arany\footnote{Az arany árfolyam forrása a  
\href{https://finance.yahoo.com/quote/GC\%3DF/history?period1=1527638400&period2=1685404800&interval=1d&filter=history&frequency=1d&includeAdjustedClose=true}{Yahoo Finance.}
}, a másik pedig a nyersolaj\footnote{Az arany árfolyam forrása a  
\href{https://finance.yahoo.com/quote/CL\%3DF/history?period1=1527638400&period2=1685404800&interval=1d&filter=history&frequency=1d&includeAdjustedClose=true}{Yahoo Finance.}
}, mivel arra számítok, hogy ezeket különböző faktorok mozgatják, kevésbé mozognak együtt, és így jól használhatók egy diverzifikált portfolió építésére. 
A mintaidőszak 2018.05.30.-tól 2023.05.26.-ig tart és 1260 megfigyelést tartalmaz mindkét eszközre. 

Napi loghozamokkal számoltam, amelyeket a következőképpen állítottam elő i eszközre, t napon:
\begin{equation}
r_{i,t}=log(\frac{P_{i,t}-P_{i,t-1}}{P_{i,t-1}} + 1).
\end{equation}

\subsection{Dinamikusan változó portfolió}
Az egyedi, dinamikusan változó portfolió képzésének alap gondolatát \cite{moskowitz2012time} Time series momentum kereskedési stratégiája adta. 
Ez egy momentum alapú kereskedési stragéia, ami abban különbözik az eredeti ... féle Cross-sectional momentumtól, hogy itt az eszközök nincsenek egymáshoz hasonlítva, csak saját múltbeli hozamaikat tekintik. 
\cite{moskowitz2012time} 55 határidős eszközön mutatta meg, hogy magas hozamot generál az a stratégia amely megveszi (eladja) azokat az eszközöket, amelyeknek az elmúlt 12 hónapos hozama pozitív (negatív) volt. 

Így a következő súlyozási szabályt alkalmazom. 
A portfolióban mindig két eszköz van $w_{t}^{A}$ és $w_{t}^{A}$ súlyokkal, amelyekre $|w_{t}^{A}| + |w_{t}^{A}| = 1$ minden időszakban. 
Legyen $R_t^i$ és $\sigma_t^i$ rendre az elmúlt $n$ nap hozama és annak szórása:
\begin{equation}
R_t^i=\sum_{T=t-n}^t r_{T}^i 
\end{equation}
Legyen $S_{i,t}=\frac{R_t^i}{\sigma_t^i}$, és legyenek a súlyok az alábbiak:
\begin{flalign}
w_t^A=& \frac{|S_{A,t}^p|}{|S_{A,t}^p| + |S_{B,t}^2|}*sign(R_t^A) \\
w_t^B=& \frac{|S_{B,t}^p|}{|S_{A,t}^p| + |S_{B,t}^2|}*sign(R_t^B).
\end{flalign}
Ezekkel a portfolió hozama t időszakban a következőképpen áll elő:
\begin{equation}
r_t^P=w_t^A*r_t^A + w_t^B*r_t^B
 \label{eq:eq1}
\end{equation}
Tehát annál nagyobb lesz a súlya egy eszköznek, minél nagyobb volt az elmúlt n napos hozam abszolútértékben, és ez utóbbinak minél kisebb volt a szórása. 

\subsection{VaR alapú modell összehasonlítás}
Az \hyperref[eq:eq1]{egyenlet (5)}-ben megadott portfolió hozama függ a napokat jelölő $n$ visszatekintési, és a $p$ kitevő paramétertől. 
Ezek változtatása különbözőféle diverzifikációhoz vezet, és ez hatással van a kockázati mutatókra is. 
A következőkben megnézem, hogy különböző $(n,p)$ paraméter kombinációk hogyan hatnak a portfolió mintaidőszakból számított VaR mutatójára. 
Az eredményt az \hyperref[table:table1]{1. Táblázat} mutatja. 
\begin{table}[!ht]
\tiny
    \centering
    \begin{tabular}{lccccc}
\textbf{1. Táblázat} &&&& &  \\
\toprule
  \multicolumn{6}{c}{\textbf{VaR értékek különböző paraméter kombinációkkal}} \\  
\midrule
	 & 1 & 2 & 3 & 4 & 5 \\ 
\cmidrule{2-6}
        5 & -2.35 & -2.43 & -2.60 & -2.62 & -2.62 \\ 
        10 & -2.39 & -2.51 & -2.63 & -2.78 & -2.77 \\ 
        20 & -2.43 & -2.71 & -2.74 & -2.81 & -2.86 \\ 
        40 & -2.38 & -2.45 & -2.57 & -2.57 & -2.58 \\ 
        60 & -2.55 & -2.79 & -2.88 & -2.97 & -3.00 \\ 
        120 & -2.69 & -2.73 & -2.84 & -2.93 & -2.98 \\ 
\bottomrule
\end{tabular}
\caption*{\tiny 
A táblázat az \hyperref[eq:eq1]{egyenlet (5)}-ben definiált portfolióhoz tartozó VaR értékeket mutatja különböző $(n,p)$ paraméter kombinációkban. 
Az n visszatekintési időszakokat a sorok, a p kitevőket az oszlopok jelölik. 
}
\label{table:table1}
\end{table}
A VaR azt a maximum százalékot mutatja, amit veszíthet a portfolió adott konfidencia intervallum mellett a következő időszakban az elmúlt időszak alapján. 
Tehát a legnagyobb lehetséges veszteséget jelzi előre. 
\hyperref[table:table1]{1. Táblázat} szerint minél nagyobb a $p$ kitevő, annál negatívabbak az értékek, ami azt jelenti, hogy nem lesz jobb a diverzifikáció, ha erősebb múltbeli napokon alapuló szignálok nagyobb súllyal érvényesülnek. 
Így a legjobb diverzifikáció az, ha a Sharpe ráta szerű szignál nincs is hatványra emelve. 
Az n paraméterek között nem lineáris kapcsolat van minden rögzített p kitevő esetében. 
Kicsivel kisebb kockázatot eredményez az 5, 10 és 40 nap, mint a 20. A legrosszabb eredményeket a 3 és 6 hónap adja. 


\section{2. Feladat}
A korrelált szimulált hozam idősorokat a következőképpen állítom elő. 
Létrehozok $Z_1$ és $Z_2$ véletlen számokból álló vektorokat inverz standard normális eloszlásból, amelynek az inputjai 0 és 1 közötti egyenletes eloszlású változók.
Jelölje $Z$ Nx2-es mátrix a két korrelálatlan oszlopvektort. 
Legyen A egy paraméter segítségvel meghatározott 2x2-es korrelációs mátrix, és D ennek Cholesky felbontásával kapott mátrix, azaz $A=D*D^T$. 
A D mátrix értelmezhető úgy, mintha A mátris "négyzetgyöke" lenne. 
Legyen $Z^c$ a korrelált idősoroknak az Nx2-es mátrix, amely a következőképpen áll elő:
\begin{equation}
Z^c=(D*Z^T)^T. 
\end{equation}
A szimulált hozamok Nx2-es mátrixa pedig a következő lesz:
\begin{equation}
R_i^S{\mu_i+Z_i^c * \sigma_i, 
\end{equation}
$Z_i^c$ az i-edik oszlop $\mu_i$ és $\sigma_i$ pedig az i-edik eszköz mintából számolt átlag hozama és szórása. 

Ezzel a módszerrel szimulálok hozam idősorokat, amelyeken olyan portfoliót értelmezek, amelyben i-edik eszközhöz tartozó fix súly a következő: 
\begin{equation}
w_i=\frac{\sigma_i^2}{\sigma_i^2 +\sigma_j^2}. 
\end{equation}

Az így kapott szimulált portfolió függ a korreláció paramétertől, ami előálltja A korrelációs mátrixot. 
Ezért megnézem, hogy a portfolióra számolt VaR értéke hogyan függ a szimuláció során alkalmazott korrelációtól. 
$-0.95$-től $0.95$-ig $0.05$-ösével növelem a korrelációt, és kirajzolom az ehhez tartozó VaR értékeket. 
Az eredményt az \hyperref[fig:fig1]{1. Ábra} mutatja. 
\begin{figure}[H]
        \centering
        %\includegraphics[height = 0.3, width = 0.6]{abra1.png}
        \includegraphics{abra1.png}
\label{fig:fig1}
\end{figure}


\section{3. Feladat}



\hyperref[fig:fig1]{2. Ábra}
\begin{figure}[H]
        \centering
        %\includegraphics[height = 0.3, width = 0.6]{abra1.png}
        \includegraphics{abra2.png}
\label{fig:fi21}
\end{figure}


\section{4. Feladat}
\section{Összefoglalás}


Although \cite{kim2016time} find that the results of \cite{moskowitz2012time} are largely driven by volatility-scaled returns, and without scaling, there is no significant difference between TSM and a buy-and-hold strategy.
In this way, I implement every analysis with both volatility-scaled and equal-weighted portfolio formation as follows:
\begin{flalign}
	r_{t+1}^{TSM_{EW}}&=\frac{2}{N}(\sum_{r_{i,t-12:t-1 \geq 0}} r_{i,t+1} - \sum_{r_{i,t-12:t-1 < 0}} r_{i,t+1}) \\
	r_{t+1}^{TSM_{VW}}&=\frac{2}{c}(\sum_{r_{i,t-12:t-1 \geq 0}}\frac{1}{\sigma_{i,t}^{r}} r_{i,t+1} - \sum_{r_{i,t-12:t-1 < 0}} \frac{1}{\sigma_{i,t}^{r}} r_{i,t+1}) 
\end{flalign}
where c is the sum of the $\frac{1}{\sigma_{i,t}^{r}}$ weights.
\newpage
\bibliography{TDK23bib.bib}
\bibliographystyle{apacite}



\end{document}




